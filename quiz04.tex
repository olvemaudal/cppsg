\documentclass[landscape]{slides}

\usepackage[margin=.5in]{geometry}

\usepackage{tabularx}
\usepackage[T1]{fontenc}
\usepackage[latin9]{inputenc}
\usepackage{listings}
\usepackage{color}
\usepackage{graphics}
\usepackage[english]{babel}

\definecolor{hellgelb}{rgb}{1,1,0.8}
\lstset{basicstyle={\tiny\ttfamily},backgroundcolor=\color{hellgelb},frame=single,numbers=left,numberstyle={\tiny\sffamily}}

\newcommand{\st}{Slide Title}

\onlyslides{0-999}

\begin{document}

%%%%%

\renewcommand{\st}{Concrete types (or value types)}

\begin{slide}

\begin{lstlisting}
class Date {
    int d, m, y;
public:
    Date(int day, int month, int year) {
        d = day;
        m = month;
        y = year;
    }
    int day() {
        return d;
    }
    int month() {
        return this->m;
    }
    int year() {
        return y;
    }
};
\end{lstlisting}

Please criticize this code.
\begin{note}
\st

\begin{tiny}
There are several things to comment here:
\begin{itemize}
\item probably not a good idea to define several variables on line 2
\item not using member initializer properly
\item possible confusion about day, day(), and d
\item a const after the argument list indicates that these functions do no modify the state of a Date object (query method)
\item a non-const methods indicates that the state of an object changes (modifier method)
\item how to handle invalid Date?
\item do we need copy constructor and assignment operator?
\item the use of this-> on line 13 is not necessary, but some prefer to write this-> for clearity.
\item it often makes sence to place data members last to empasize the funtions providing the public user interface [p234]
\end{itemize}

\end{tiny}
\end{note}

\end{slide}

%%%%%

\renewcommand\st{mutable, name collisions}
\begin{slide}

\tiny{foo.hpp}
\begin{lstlisting}
#include <iostream>

class Foo {
    int value;
    int read_counter;
public:
    Foo(int v) : value(v), read_counter(0) { }
    int value();
    void debug_print_read_counter();
};
\end{lstlisting}

\begin{tabular*}{\textwidth}{@{\extracolsep{\fill}}lr}

\begin{minipage}[t]{.52\linewidth}
\tiny{foo.cpp}
\begin{lstlisting}
#include "foo.hpp"

int Foo::value() const {
    read_counter++;
    return value;
}

void Foo::debug_print_read_counter() {
    std::cerr << "read_counter = " 
              << read_counter << std::endl;
}
\end{lstlisting}
\end{minipage}
&
\begin{minipage}[t]{.43\linewidth}
\tiny{main.cpp}
\begin{lstlisting}
#include "foo.hpp"

int main() {
    Foo f(42);
    f.debug_print_read_counter();
    f.value();
    f.value();
    f.debug_print_read_counter();
}
\end{lstlisting}
\end{minipage}
\end{tabular*}

\

\begin{lstlisting}[numbers=none]
g++ foo.cpp main.cpp && ./a.out  
\end{lstlisting}

This code does not compile and it smells bad. How can you fix it? Please criticize.

\begin{note}
\st
\begin{tiny}

\begin{verbatim}
- iosfwd
- conflicting redeclaration of Foo::value() and Foo::value
- consider underscore postfixing of member variables
- declaration on foo.hpp:8 and definition on foo.cpp:3 does not match
- logical constness (p241). consider making read_counter mutable
- debug_print_read_counter should be const, indicating that it is a query
- there is a cohesion problem with Foo.cpp
\end{verbatim}

Once fixed, it might print:
\begin{verbatim}
read_counter = 0
read_counter = 2
\end{verbatim}

\end{tiny}
\end{note}

\end{slide}

%%%%%

\renewcommand\st{new[]/delete[], the rule of three}
\begin{slide}

\begin{lstlisting}
#include <iostream>

template<typename T> void p(T x) { std::cout << x; }

struct B {
    B() { p('B'); }
    ~B() { p('b'); }
};

class A {
    B * v;
public:
    A() { p('A'); v = new B[3]; }
    ~A() { p('a'); delete v; }
};

int main() {
    A a1;
    A a2 = a1;
    A a3;
    a3 = a1;
}
\end{lstlisting}

What might happen if you try to compile, link and run this program? Please criticize.

\begin{note}
\st
\begin{tiny}
On my machine I get:
\begin{verbatim}
g++ -Wall scratch.cpp && ./a.out
ABBBABBBababab
\end{verbatim}

\begin{itemize}
\item if you new an array you must delete an array
\item the rule of three, if you need a destructor, you probably need a copy constructor and copy assignment operator
\item deleting something twice is a serious error; the behaviour is undefined and most likely disastrous
\item should the destructor of A be virtual?
\end{itemize}

\end{tiny}
\end{note}

\end{slide}

%%%%%

\renewcommand\st{initialization of objects}
\begin{slide}

\begin{lstlisting}
#include <iostream>

template<typename T> void p(T x) { std::cout << x; }

struct A {
    A() { p('A'); }
    ~A() { p('a'); }
};

struct B {
    B() { p('B'); }
    ~B() { p('b'); }
};

class C {
    A a;
    B b;
public:
    C() : b(), a() {}
    C(int) {}
};

int main() {
    C c1;
    C c2(4);
}
\end{lstlisting}

what might happen if you try to compile, link and run this program?

\begin{note}
\st

\begin{tiny}
On my machine I get:
\begin{verbatim}
g++ -Wall scratch.cpp && ./a.out
scratch.cpp: In constructor 'C::C()':
scratch.cpp:17: warning: 'C::b' will be initialized after
scratch.cpp:16: warning:   'A C::a'
scratch.cpp:19: warning:   when initialized here
ABABbaba
\end{verbatim}

Without -Wall, no warning is printed. \

If a member constructor needs no argument, the member need not to be mentioned in the member initializer list.
\
what about native type members?
\end{tiny}
\end{note}

\end{slide}

%%%%%

\renewcommand\st{using member initializer}
\begin{slide}

\begin{lstlisting}
#include <iostream>

template<typename T> void p(T x) { std::cout << x; }

struct A {
    int value;
    A() { p(0); }
    A(int v) : value(v) { p(1); }
    ~A() { p(2); }
    A(const A & a) { p(3); }
    void operator=(const A & a) { p(4); }
};

struct B {
    A a;
    B(int v) { p('B'); a=42; p('b'); }
};

struct C {
    A a;
    C(int v) : a(v) { p('C'); p('c'); }
};

int main() {
    p('-');
    B b(42);
    p('-');
    C c(42);
    p('-');
}
\end{lstlisting}

What might happen if you try to compile, link and run this program?

\begin{note}
\st
\begin{tiny}
\begin{verbatim}
g++ -Wall scratch.cpp && ./a.out
-0B142b-1Cc-22
\end{verbatim}
\end{tiny}
\end{note}

\end{slide}

%%%%%

\renewcommand\st{placement}
\begin{slide}

\begin{lstlisting}
#include <iostream>

template<typename T> void p(T x) { std::cout << x; }

struct X {
    int a;
    int b;
    X() : a(1), b(2) {}
};

struct Y {
    int c;
    int d;
    Y() : c(3), d(4) {}
};

int main() {
    X x;
    new (&x) Y;
    p(x.a);
    p(x.b);
}
\end{lstlisting}

what might happen if you try to compile, link and run this program?

\begin{note}
\st
\begin{tiny}
\begin{verbatim}
g++ -Wall scratch.cpp && ./a.out
34
\end{verbatim}

\end{tiny}
\end{note}

\end{slide}

%%%%%

\renewcommand\st{temporary objects}
\begin{slide}

\begin{lstlisting}
#include <iostream>

template<typename T> void p(T x) { std::cout << x; }

struct X {
    int value;
    X(int v) : value(v) { p('X'); p(v); }
    ~X() { p('x'); }
    X(const X & x) : value(x.value) { p(2); }
    X operator+(const X & x) { p(3); return this->value + x.value; }
};

int main() {
    X a(4);
    p('-');
    p( (a + a).value );
    p('-');
    X b = (a + a);
    p('-');
    p(b.value);
    p('-');
}
\end{lstlisting}

what might happen if you try to compile, link and run this program?

\begin{note}
\st
\begin{tiny}
\begin{verbatim}
g++ -Wall scratch.cpp && ./a.out
X4-3X88x-3X8-8-xx
\end{verbatim}

\end{tiny}
\end{note}

\end{slide}

%%%%%

\renewcommand\st{implicit conversion}
\begin{slide}

\begin{lstlisting}
#include <iostream>

template<typename T> void p(T x) { std::cout << x; }

struct X {
    X(int v) { p('X'); p(v); }
    ~X() { p('x'); }
};

void foo(X x) {
    p('F');
}

int main() {
    foo(4);
}
\end{lstlisting}

what might happen if you try to compile, link and run this program?

\begin{note}
\st
\begin{tiny}
\begin{verbatim}
g++ -Wall scratch.cpp && ./a.out
X4Fx
\end{verbatim}
\end{tiny}
\end{note}

\end{slide}

%%%%%

\renewcommand\st{conversion operator}
\begin{slide}

\begin{lstlisting}
#include <iostream>

template<typename T> void p(T x) { std::cout << x; }

struct X {
    int value;
    X(int v) : value(v) { }
    operator int() const { return value; }
};

void foo(int v) {
    p(v);
}

int main() {
    X x = 42;
    foo(x);
}
\end{lstlisting}

what might happen if you try to compile, link and run this program?

\begin{note}
\st
\begin{tiny}
\begin{verbatim}
g++ -Wall scratch.cpp && ./a.out
42
\end{verbatim}

What happens if you put explicit in front of line 7? What would you then need to write on line 16?
\end{tiny}
\end{note}

\end{slide}

%%%%%

\renewcommand\st{functor}
\begin{slide}

\begin{lstlisting}
#include <iostream>
#include <vector>

class X {
    int value;
public:
    X(int v) : value(v) {}
    void operator()(int & i) const {
        i += value;
    }
};

void p(int & i) {
    std::cout << i << std::endl;
}

int main() {
    std::vector<int> v;
    v.push_back(1);
    v.push_back(2);
    v.push_back(3);
    for_each(v.begin(), v.end(), X(42));
    for_each(v.begin(), v.end(), p);
}
\end{lstlisting}

what might happen if you try to compile, link and run this program?

\begin{note}
\st
\begin{tiny}
\begin{verbatim}
g++ -Wall scratch.cpp && ./a.out
43
44
45
\end{verbatim}

\end{tiny}
\end{note}

\end{slide}

%%%%%

%% template

%% \renewcommand\st{�x.x, title}
%% \begin{slide}
%% 
%% \begin{lstlisting}
%% 
%% \end{lstlisting}
%% 
%% what might happen if you try to compile, link and run this program?
%% 
%% \begin{note}
%% \st
%% \begin{tiny}
%% \begin{verbatim}
%% 
%% \end{verbatim}
%% 
%% \end{tiny}
%% \end{note}
%% 
%% \end{slide}

\end{document}
