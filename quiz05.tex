\documentclass[landscape]{slides}

\usepackage[margin=.5in]{geometry}

\usepackage{tabularx}
\usepackage[T1]{fontenc}
\usepackage[latin9]{inputenc}
\usepackage{listings}
\usepackage{color}
\usepackage{graphics}
\usepackage[english]{babel}

\definecolor{hellgelb}{rgb}{1,1,0.8}
\lstset{basicstyle={\tiny\ttfamily},backgroundcolor=\color{hellgelb},frame=single,numbers=left,numberstyle={\tiny\sffamily},firstnumber=auto}

\newcommand{\st}{Slide Title}

\onlyslides{0-999}

\begin{document}

%%%%%

\renewcommand\st{�12.2, Derived classes}
\begin{slide}

\begin{lstlisting}
class X;

class Y : public X {
    int b;
};

class X {
    int a;
};
\end{lstlisting}

What might happen if you try to compile this code?

\begin{note}
\st
\begin{tiny}

On my machine I get:

\begin{verbatim}
g++ scratch.cpp 
scratch.cpp:3: error: invalid use of undefined type 'struct X'
scratch.cpp:1: error: forward declaration of 'struct X'
\end{verbatim}

[p304] Using a class as a base is equivalent to declaring an (unnamed) object of
that class. Consequently, a class must be defined in order to be used as a base.

\end{tiny}
\end{note}

\end{slide}

%%%%%

\renewcommand\st{�12.2.2 Constructor}
\begin{slide}

\begin{lstlisting}
struct Foo {
    Foo(int a, int b) : a_(a), b_(b) {}
    int a_;
    int b_;
};

struct Bar : Foo {
    Bar(int a, int b, int c) : a_(a), b_(b), c_(c) {}
    int c_;
};
\end{lstlisting}

This code does not compile. Why not? How to fix? 

\begin{note}
\st

\begin{tiny}
\begin{verbatim}
g++ scratch.cpp 
scratch.cpp: In constructor 'Bar::Bar(int, int, int)':
scratch.cpp:8: error: class 'Bar' does not have any field named 'a_'
scratch.cpp:8: error: class 'Bar' does not have any field named 'b_'
scratch.cpp:8: error: no matching function for call to 'Foo::Foo()'
scratch.cpp:2: note: candidates are: Foo::Foo(int, int)
scratch.cpp:1: note:                 Foo::Foo(const Foo&)
\end{verbatim}

[p306] A derived class constructor can specify initializers for its own members and
immediate bases only; it cannot directly initialize members of a base.

To fix this you might consider something like:
\begin{verbatim}
struct Bar : Foo {
    Bar(int a, int b, int c) : Foo(a,b), c_(c) {}
    int c_;
};
\end{verbatim}

\end{tiny}

\end{note}

\end{slide}

%%%%%

\renewcommand\st{�12.2.2, Constructors and Destructors}
\begin{slide}

\begin{lstlisting}
#include <iostream>

template<typename T> void p(T x) { std::cout << x; }

struct Foo {
    Foo() { p(1); }
    ~Foo() { p(2); }
};

struct Bar : Foo {
    Bar() { p(3); }
    ~Bar() { p(4); }
};

struct Gaz : Bar {
    Gaz() { p(5); }
    ~Gaz() { p(6); }
};

int main() {
    p('-');
    Foo f;
    p('-');
    Bar b;
    p('-');
    Gaz g;
    p('-');
}
\end{lstlisting}

What will this code print out?

\begin{note}
\st

\begin{tiny}
\begin{verbatim}
g++ scratch.cpp && ./a.out
-1-13-135-642422
\end{verbatim}
\end{tiny}

[p307] Class objets are constructed from the bottom up: first the
base, then the members, and then the derived class itself. They are
destroyed in the opposite order: first the derived class itself, then
the members, and then the base. Members and bases are constructed in
order of declaration in the class and destroyed in the reverse order.

\end{note}

\end{slide}

%%%%%

\renewcommand\st{�12.2.4, Class Hierarchies}
\begin{slide}

\begin{lstlisting}
#include <iostream>

template<typename T> void p(T x) { std::cout << x; }

struct A {
    A() { p(1); }
    ~A() { p(2); }
};

struct B : A {
    B() { p(3); }
    ~B() { p(4); }
};

struct C : A, B {
    C() { p(5); }
    ~C() { p(6); }
};

struct D {
    D() { p(7); }
    ~D() { p(8); }
};

struct E : D, B, C {
    E() { p(9); }
    ~E() { p(0); }
};

int main() {
    E e;
    p('-');
}
\end{lstlisting}

What might happen if you try to compile, link and run this code?

\begin{note}
\st

\begin{tiny}
\begin{verbatim}
g++ scratch.cpp && ./a.out
scratch.cpp:15: warning: direct base 'A' inaccessible in 'C' due to ambiguity
scratch.cpp:25: warning: direct base 'B' inaccessible in 'E' due to ambiguity
71311359-06422428
\end{verbatim}
\end{tiny}

\end{note}

\end{slide}

%%%%%

\renewcommand\st{Public/Protected/Private Inheritance}
\begin{slide}

\begin{lstlisting}
class A {
    // ...
};

class B : private A {
    // ...
};

class C : protected A {
    // ...
};

class D : public A {
    // ...
};

int main() {
    A * a1 = new B();
    A * a2 = new C();
    A * a3 = new D();
}
\end{lstlisting}

Consider this code. What might happen when you try to compile
this code? What does private inheritance mean? When should you
use protected inheritance?

\begin{note}
\st

\begin{tiny}
\begin{verbatim}
g++ scratch.cpp && ./a.out
scratch.cpp: In function 'int main()':
scratch.cpp:18: error: 'A' is an inaccessible base of 'B'
scratch.cpp:19: error: 'A' is an inaccessible base of 'C'
\end{verbatim}
\end{tiny}

What about using reinterpret_cast<A*>?
\end{note}

\end{slide}

%%%%%

\renewcommand\st{Virtual Destructor}
\begin{slide}

\begin{lstlisting}
#include <iostream>

template<typename T> void p(T x) { std::cout << x; }

class A {
    char * s;
public:
    A() { p('A'); s = new char[1024]; }
    ~A() { p('a'); delete[] s; }
};

class Foo {
    A a;
public:
    virtual void print() = 0;
};

class Bar : public Foo {
    A a;
public:
    virtual void print() { p(1); };
};

int main() {
    p('-');
    Foo * f = new Bar;
    p('-');
    f->print();
    p('-');
    delete f;
    p('-');
}
\end{lstlisting}

What will this code print out? Please criticize.

\begin{note}
\st

\begin{tiny}
\begin{verbatim}
g++ -Wall scratch.cpp && ./a.out
scratch.cpp:12: warning: 'class Foo' has virtual functions but non-virtual destructor
scratch.cpp:18: warning: 'class Bar' has virtual functions but non-virtual destructor
-AA-1-a-
\end{verbatim}

We create two objects of type A, but destroy only one. How to fix? eg, virtual ~Foo() {}

Will a virtual Foo destructor fix the warning on line 18?

Should A have a virtual destructor? No. Same problem can happen with A, but without
virtual members it is kind of obvious that it was not designed for polymorphism.

\end{tiny}

\end{note}

\end{slide}

%%%%%

\renewcommand\st{Where to put virtual destructors}
\begin{slide}

\begin{tabular*}{\textwidth}{@{\extracolsep{\fill}}lr}

\begin{minipage}[t]{.5\linewidth}

\begin{lstlisting}[name=A]
#include <iostream>

template<typename T> void p(T x) { 
    std::cout << x; 
}

struct A {
    A() { p('A'); }
    ~A() { p('a'); }
};

struct B : A {
    B() { p('B'); }
    ~B() { p('b'); }
};

class Foo {
public:
    virtual A * create() const { 
        p(1); 
        return new A(); 
    }
};

class Bar : public Foo {
public:
    virtual B * create() const { 
        p(2); 
        return new B(); 
    }
};

\end{lstlisting}
\end{minipage}
&
\begin{minipage}[t]{.38\linewidth}
\begin{lstlisting}[name=A]
int main() {
    const Foo & f1 = Foo();
    A * a1 = f1.create();
    delete a1;
    p('-');

    const Foo & f2 = Bar();
    A * a2 = f2.create(); 
    delete a2;
    p('-');

    const Bar & f3 = Bar();
    B * a3 = f3.create(); 
    delete a3;
    p('-');
}
\end{lstlisting}
\end{minipage}
\end{tabular*}

What might happen if you try to compile, link and run this program?

\begin{note}
\st

\begin{tiny}
\begin{verbatim}
g++ -Wall scratch.cpp && ./a.out
scratch.cpp:17: warning: 'class Foo' has virtual functions but non-virtual destructor
scratch.cpp:25: warning: 'class Bar' has virtual functions but non-virtual destructor
1Aa-2ABa-2ABba-
\end{verbatim}

What will happen if we add a virtual destructor to Foo? Nothing

What will happen if we add virtual to A destructor? 1Aa-2ABba-2ABba-

\end{tiny}

\end{note}

\end{slide}

%%%%%

\renewcommand\st{Templates}
\begin{slide}

\begin{tabular*}{\textwidth}{@{\extracolsep{\fill}}lr}

\begin{minipage}[t]{.55\linewidth}

\begin{lstlisting}[name=Templates]
#include <iostream>

struct A {
    void func() { std::cout << 'A'; }
};

struct B {
    void func() { std::cout << 'B'; }
};

struct C {
    void run() { std::cout << 'C'; }
};

template<typename T> class Foo {
public:
    Foo(const T & t) : value(t) { }
    void run();
private:
    T value;
};

template<typename T> void Foo<T>::run() {
    value.func();
}

\end{lstlisting}
\end{minipage}
&
\begin{minipage}[t]{.38\linewidth}
\begin{lstlisting}[name=Templates]
int main() {
    A a;
    Foo<A> fa(a);
    fa.run();
    
    B b;
    Foo<B> fb(b);
    fb.run();
    
    C c;
    Foo<C> fc(c);
    c.run();
}
\end{lstlisting}
\end{minipage}
\end{tabular*}

What might happen if you try to compile, link and run this program?

\begin{note}
\st

\begin{tiny}
\begin{verbatim}
g++ -Wall scratch.cpp && ./a.out
ABC
\end{verbatim}

[p327] A template depends only on the properties that it actually uses
from its parameter types and does not require different types used as 
arguments to be explicitly related. In particular, the argument types
used for a template need not be from a single inheritance hierarchy.

\end{tiny}

\end{note}

\end{slide}

%%%%%

\renewcommand\st{Template Parameters}
\begin{slide}

\begin{lstlisting}
#include <iostream>

template<typename T, int maxsize> class Foo {
    T vector_[maxsize];
public:
    int size() {
        return sizeof(vector_) / sizeof(T);
    }
    // ...
};

int main() {
    Foo<int, 4> a;
    std::cout << a.size();

    int sz=2;
    Foo<char, sz> b;
    std::cout << b.size();
}
\end{lstlisting}

What might happen if you try to compile, link and run this program?

\begin{note}
\st

\begin{tiny}
\begin{verbatim}
g++ -Wall scratch.cpp && ./a.out
scratch.cpp: In function 'int main()':
scratch.cpp:17: error: 'sz' cannot appear in a constant-expression
scratch.cpp:17: error: template argument 2 is invalid
scratch.cpp:17: error: invalid type in declaration before ';' token
scratch.cpp:18: error: request for member 'size' in 'b', which is of non-class type 'int'
\end{verbatim}

You can fix this by adding const on line 16. Then you get: 42

Note that line 7 could have been replaced by "return maxsize;". This partly
explains why maxsize must be a const.
\end{tiny}

\end{note}

\end{slide}

%%%%%

\renewcommand\st{Function Templates}
\begin{slide}

\begin{lstlisting}
#include <iostream>

template<typename T, T threshold> bool gt(T t) {
    std::cout << t << '>' << threshold;
    return t > threshold;
}

int main() {
    std::cout << std::boolalpha << "=" << gt<int,5>(3);
}
\end{lstlisting}

What might happen if you try to compile, link and run this program?

\begin{note}
\st

\begin{tiny}
\begin{verbatim}
g++ -Wall scratch.cpp && ./a.out
3>5=false
\end{verbatim}
\end{tiny}

\end{note}

\end{slide}

%%%%%

\renewcommand\st{Implicit and Explicit Specification}
\begin{slide}

\begin{lstlisting}
#include <iostream>
#include <sstream>

template<typename T> bool istrue(T t) {
    std::stringstream s;
    s << t;
    return s.str() == "42";
}

int main() {
    std::cout << std::boolalpha;

    int t = 42;
    std::cout << istrue(t) << std::endl;

    float f = 43;
    std::cout << istrue(f) << std::endl;

    char * s = "42";
    std::cout << istrue(s) << std::endl;

    char c = 42;
    std::cout << istrue(c) << std::endl;
}
\end{lstlisting}

What might happen if you try to compile, link and run this program?

\begin{note}
\st

\begin{tiny}
\begin{verbatim}
g++ -Wall scratch.cpp && ./a.out
true
false
true
false
\end{verbatim}

How can we make line 23 print out true? Eg, by doing explicit specification
of the template argument

\begin{verbatim}
std::cout << istrue<int>(c) << std::endl;
\end{verbatim}

\end{tiny}

\end{note}

\end{slide}

%%%%%

\renewcommand\st{Order of specializations}
\begin{slide}

\begin{lstlisting}
#include <iostream>
#include <sstream>

template<typename T> void foo(T t) {
    std::cout << 'a';
}

void foo(float t) {
    std::cout << 'b';
}

void foo(long t) {
    std::cout << 'c';
}

int main() {
    foo(42);
    foo("42");
    foo(42L);
    foo(42.3);
}
\end{lstlisting}

What might happen if you try to compile, link and run this program?

\begin{note}
\st

\begin{tiny}
\begin{verbatim}
g++ -Wall scratch.cpp && ./a.out
aaca
\end{verbatim}
\end{tiny}

\end{note}

\end{slide}

%%%%%

\renewcommand\st{Template Function Specializations}
\begin{slide}

\begin{lstlisting}
#include <iostream>

template<typename T> void foo(T t) {
    std::cout << 1;
}

template<typename T> void foo(T * t) {
    std::cout << 2;
}

template<> void foo(char) {
    std::cout << 4;
}

int main() {
    int a;
    foo(a);
    foo(&a);
    char b;
    foo(b);
    foo(&b);
}
\end{lstlisting}

What might happen if you try to compile, link and run this program?

\begin{note}
\st

\begin{tiny}
\begin{verbatim}
g++ -Wall scratch.cpp && ./a.out
1242
\end{verbatim}
\end{tiny}

\end{note}

\end{slide}

%%%%%

%% \renewcommand\st{�x.x, title}
%% \begin{slide}
%% 
%% \begin{lstlisting}
%% 
%% \end{lstlisting}
%% 
%% what might happen if you try to compile, link and run this program?
%% 
%% \begin{note}
%% \st
%%
%% \begin{tiny}
%% \begin{verbatim}
%% 
%% \end{verbatim}
%% \end{tiny}
%%
%% \end{note}
%% 
%% \end{slide}

\end{document}
