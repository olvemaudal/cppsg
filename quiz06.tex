\documentclass[landscape]{slides}

\usepackage[margin=.5in]{geometry}

\usepackage{tabularx}
\usepackage[T1]{fontenc}
\usepackage[latin9]{inputenc}
\usepackage{listings}
\usepackage{color}
\usepackage{graphics}
\usepackage[english]{babel}

\definecolor{hellgelb}{rgb}{1,1,0.8}
\lstset{basicstyle={\tiny\ttfamily},backgroundcolor=\color{hellgelb},frame=single,numbers=left,numberstyle={\tiny\sffamily},firstnumber=auto}

\newcommand{\st}{Slide Title}

\onlyslides{0-999}

\begin{document}

%%%%%

\renewcommand\st{�14.2, Grouping of Exceptions}
\begin{slide}

\begin{lstlisting}
#include <iostream>

template<typename T> void p(T x) { std::cout << x; }

class A { };
class B : public A { };
class C : public A { };

void foo() {
    p("1");
    throw C();
}

int main() {
    try {
        p("2");
        foo();
        p("3");
    } catch(B) {
        p("4");
    } catch(C) {
        p("5");
    } catch(A) {
        p("6");
    }
    p("7");
}
\end{lstlisting}

What might happen if you try to compile this code?

\begin{note}
\st
\begin{tiny}

\begin{verbatim}
g++ scratch.cpp && ./a.out
2157
\end{verbatim}

\end{tiny}
\end{note}

\end{slide}

%%%%%

\renewcommand\st{�15.3.2.2, Using-Declarations and Access Control}
\begin{slide}

\begin{lstlisting}
#include <iostream>

template<typename T> void p(T x) { std::cout << x; }

class A {
protected:
    void foo() { p("foo"); };
};

class B : virtual public A {
public:
    void bar() { p("bar"); };
};
    
int main() {
    B b;
    b.foo();
}
\end{lstlisting}

what might happen if you try to compile, link and run this program?

\begin{note}
\st

\begin{tiny}
I get:
\begin{verbatim}
g++ -Wall scratch.cpp && ./a.out
scratch.cpp: In function 'int main()':
scratch.cpp:7: error: 'void A::foo()' is protected
scratch.cpp:17: error: within this context
\end{verbatim}
How con you make this code compile and print "foo"? Eg, by adding "using A::foo;" in
the public part of B.
\end{tiny}

\end{note}

\end{slide}

%%%%%

\renewcommand\st{�14.2, Derived Exceptions}
\begin{slide}

\begin{lstlisting}
#include <iostream>

template<typename T> void p(T x) { std::cout << x; }

class A {
public:
    virtual void boo() { p("A"); };
};

class B : public A {
public:
    virtual void boo() { p("B"); };
};

int main() {
    try {
        throw B();
    } catch( A a ) {
        a.boo();
    }
}
\end{lstlisting}

what might happen if you try to compile, link and run this program?

\begin{note}
\st

\begin{tiny}
\begin{verbatim}
g++ -Wall scratch.cpp && ./a.out
scratch.cpp:5: warning: 'class A' has virtual functions but non-virtual destructor
scratch.cpp:10: warning: 'class B' has virtual functions but non-virtual destructor
A
\end{verbatim}

How to make this code write "B"? Catch by reference

\end{tiny}

\end{note}

\end{slide}

%%%%%

\renewcommand\st{14.3.1 Re-Throw}
\begin{slide}

\begin{lstlisting}
#include <iostream>

template<typename T> void p(T x) { std::cout << x; }

class A { };

int main() {
    p("0");
    try {
        p("1");
        try {
            p("2");
            throw A();
            p("3");
        } catch(...) {
            p("4");
            throw;
            p("5");
        }
    } catch(A a) {
        p("6");
        throw;
        p("7");
    }
    p("8");
}
\end{lstlisting}

what might happen if you try to compile, link and run this program?

\begin{note}
\st

\begin{tiny}
\begin{verbatim}
g++ -Wall scratch.cpp && ./a.out
terminate called after throwing an instance of 'A'
01246
\end{verbatim}
\end{tiny}

\end{note}

\end{slide}

%%%%%

\renewcommand\st{�14.4.1, RAII}
\begin{slide}

\begin{lstlisting}
#include <iostream>

template<typename T> void p(T x) { std::cout << x; }

class B {
public:
    B() { p("B"); }
    ~B() { p("b"); }
};

class A {
    B * v;
public:
    A() { p("A"); v = new B[3]; throw "x"; }
    ~A() { p("a"); delete[] v; }
};

int main() {
    try {
        p("0");
        A a;
        p("1");
    } catch( const char * s ) {
        p(s);
    }
}
\end{lstlisting}

what might happen if you try to compile, link and run this program?

\begin{note}
\st

\begin{tiny}
\begin{verbatim}
0ABBBx
\end{verbatim}
What if you remove the throw on line 14?
\begin{verbatim}
0ABBB1abbb
\end{verbatim}
\end{tiny}

\end{note}

\end{slide}

%%%%%

\renewcommand\st{�14.4.6.1, Exceptions and Member Initialization}
\begin{slide}

\begin{lstlisting}
#include <iostream>

template<typename T> void p(T x) { std::cout << x; }

class B {
public:
    B() { p("B"); throw "x"; }
    ~B() { p("b"); }
};

class A {
    B b;
public:
    A() try : b() { p("A"); } catch(...) { p("y"); }
    ~A() { p("a"); }
};

int main() {
    try {
        p("0");
        A a;
        p("1");
    } catch( const char * s ) {
        p(s);
    }
}
\end{lstlisting}

what might happen if you try to compile, link and run this program?

\begin{note}
\st

\begin{tiny}
\begin{verbatim}
g++ -Wall scratch.cpp && ./a.out
0Byx
\end{verbatim}
\end{tiny}

\end{note}

\end{slide}

%%%%%

\renewcommand\st{�14.6, Exception Specification}
\begin{slide}

\begin{lstlisting}
#include <iostream>

template<typename T> void p(T x) { std::cout << x; }

void foo() throw() {
    throw "x";
};

int main() {
    try {
        p("0");
        foo();
        p("1");
    } catch( const char * s ) {
        p(s);
    }
}
\end{lstlisting}

what might happen if you try to compile, link and run this program?

\begin{note}
\st

\begin{tiny}
\begin{verbatim}
g++ -Wall scratch.cpp && ./a.out
terminate called after throwing an instance of 'char const*'
0
\end{verbatim}
\end{tiny}

\end{note}

\end{slide}

%%%%%

\renewcommand\st{�15.2.2, Inheritance and Using-Declarations}
\begin{slide}

\begin{lstlisting}
#include <iostream>

template<typename T> void p(T x) { std::cout << x; }

struct B {
    void foo(int) { p(1); }
};

struct C : B {
    void foo(float) { p(2); }
};

int main() {
    C c;
    c.foo(42);
}
\end{lstlisting}

what might happen if you try to compile, link and run this program?

\begin{note}
\st

\begin{tiny}
\begin{verbatim}
g++ -Wall scratch.cpp && ./a.out
2
\end{verbatim}
How to make this print 1? Add "using B::foo" after line 9.
\end{tiny}

\end{note}

\end{slide}

%%%%%

\renewcommand\st{�15.2.4, Virtual Base Classes}
\begin{slide}

\begin{lstlisting}
#include <iostream>

template<typename T> void p(T x) { std::cout << x; }

struct A {
    A() { p('A'); }
};

struct B : A {
    B() { p('B'); }
};

struct C : virtual A {
    C() { p('C'); }
};

struct D : B, C {
    D() { p('D'); }
};
    
int main() {
    D d;
}
\end{lstlisting}

what might happen if you try to compile, link and run this program?

\begin{note}
\st

\begin{tiny}
\begin{verbatim}
g++ -Wall scratch.cpp && ./a.out
AABCD
\end{verbatim}
\end{tiny}
What if we add "virtual" on line 9? ABCD
What if we remove "virtual" on line 13? ABACD


\end{note}

\end{slide}

%%%%%

\renewcommand\st{�x.x, title}
\begin{slide}

\begin{lstlisting}
#include <iostream>
#include <typeinfo>

template<typename T> void p(T x) { std::cout << x; }

struct A {
    virtual void foo() { p("A"); }
};

struct B : A {
    virtual void foo() { p("B"); }
};

struct C : A {
    virtual void foo() { p("C"); }
};

int main() {
    A * b = new B;
    A * c = new C;
    b = c;
    b->foo();
    b = dynamic_cast<B*>(c);
    b->foo();
}
\end{lstlisting}

what might happen if you try to compile, link and run this program?

\begin{note}
\st

\begin{tiny}
I got:
\begin{verbatim}
g++ -Wall scratch.cpp && ./a.out
scratch.cpp:6: warning: 'struct A' has virtual functions but non-virtual destructor
scratch.cpp:10: warning: 'struct B' has virtual functions but non-virtual destructor
scratch.cpp:14: warning: 'struct C' has virtual functions but non-virtual destructor

Compilation exited abnormally with code 138 at Fri Oct 26 02:14:06
\end{verbatim}
Why does it crash? Because b is 0 on line 24.

I did expect that it printed a "C", but std is not flushed so it is stuck in the buffer.
Ass a std::flush to line 4 and it should print "C" before crashing. 

Consider line 23 which gives b = 0. Is there a useful way of using
dynamic cast like this? To determine if some object is of a particular type.

What if we replace line 23 with a static cast or just plain old cast? Then you get "CC"

What if you remove "virtual" on line 7,11,15? Then you get a compile error:
\begin{verbatim}
g++ -Wall scratch.cpp && ./a.out
scratch.cpp: In function 'int main()':
scratch.cpp:23: error: cannot dynamic_cast 'c' (of type 'struct A*') to type 'struct B*' 
                (source type is not polymorphic)
\end{verbatim}

That is quite handy.

\end{tiny}

\end{note}

\end{slide}

%%%%%

\renewcommand\st{�15.4.4, Typeid and Extended Type Info}
\begin{slide}

\begin{lstlisting}
#include <iostream>
#include <typeinfo>

struct Foo {
};

struct Bar : Foo {
};

struct Gaz : Foo {
};

int main() {
    Foo * b = new Bar;
    Foo * c = new Gaz;
    std::cout << typeid(*b).name();
    std::cout << typeid(*c).name();
}
\end{lstlisting}

what might happen if you try to compile, link and run this program?

\begin{note}
\st

\begin{tiny}
I get:
\begin{verbatim}
g++ -Wall scratch.cpp && ./a.out
3Foo3Foo
\end{verbatim}

How can you get this code write out "3Bar3Gaz"? Add a dummy virtual 
function to Foo.

\end{tiny}

\end{note}

\end{slide}

%%%%%
%% 
%% \renewcommand\st{�x.x, title}
%% \begin{slide}
%% 
%% \begin{lstlisting}
%% 
%% \end{lstlisting}
%% 
%% what might happen if you try to compile, link and run this program?
%% 
%% \begin{note}
%% \st
%% 
%% \begin{tiny}
%% \begin{verbatim}
%% 
%% \end{verbatim}
%% \end{tiny}
%% 
%% \end{note}
%% 
%% \end{slide}

\end{document}
