\documentclass[landscape]{slides}

\usepackage[margin=.5in]{geometry}

\usepackage{tabularx}
\usepackage[T1]{fontenc}
\usepackage[latin9]{inputenc}
\usepackage{listings}
\usepackage{color}
\usepackage{graphics}
\usepackage[english]{babel}
\usepackage{underscore}

\definecolor{hellgelb}{rgb}{1,1,0.8}
\lstset{basicstyle={\tiny\ttfamily},backgroundcolor=\color{hellgelb},frame=single,numbers=left,numberstyle={\tiny\sffamily},firstnumber=auto}

\newcommand{\st}{Slide Title}

\onlyslides{0-999}

\begin{document}

%%%%%

\renewcommand\st{�x.x, title}
\begin{slide}

\begin{lstlisting}
#include <iostream>
#include <string>
#include <algorithm>

int main() {
    std::string s("kulturuke");
    std::cout << s << std::endl;
    while( next_permutation(s.begin(),s.end()) )
        std::cout << s << std::endl;
}
\end{lstlisting}

what might happen if you try to compile, link and run this program?

\begin{note}
\st

\begin{tiny}
I get:
\begin{verbatim}
g++ -Wall scratch.cpp && ./a.out
kulturuke
kultuuekr
kultuuerk
... 21721 more
uuutrlekk
uuutrlkek
uuutrlkke
\end{verbatim}

What if you sort the string before entering the do loop? I get
30240 lines:
\begin{verbatim}
g++ -Wall scratch.cpp && ./a.out
ekklrtuuu
ekklrutuu
ekklruutu
... 
uuutrlekk
uuutrlkek
uuutrlkke
\end{verbatim}

Is string a regular container?

\end{tiny}

\end{note}

\end{slide}

%%%%%

\renewcommand\st{�x.x, title}
\begin{slide}

\begin{lstlisting}
#include <iostream>
#include <string>

int main() {
    std::string s("Hello World!");
    std::wstring ws(s);
    std::cout << ws << std::endl;
}
\end{lstlisting}

This code does not compile. How to fix?
\begin{note}
\st

\begin{tiny}

I get errors on both line 6 and 7. 
\begin{verbatim}
g++ -Wall scratch.cpp && ./a.out
scratch.cpp: In function 'int main()':
scratch.cpp:6: error: no matching function for call to 
  'std::basic_string<wchar_t, std::char_traits<wchar_t>, 
  std::allocator<wchar_t> >::basic_string(std::string&)'
...
scratch.cpp:7: error: no match for 'operator<<' in 'std::cout << ws'
...
\end{verbatim}

Here is one way to "fix":
\begin{verbatim}
#include <iostream>
#include <string>

int main() {
    std::string s("Hello World!");
    std::wstring ws(s.begin(), s.end());
    std::wcout << ws << std::endl;
}
\end{verbatim}

[p586],[p609]

\end{tiny}


\end{note}

\end{slide}

%%%%%

\renewcommand\st{�x.x, title}
\begin{slide}

\begin{lstlisting}
#include <iostream>
#include <string>

int main() {
    std::string s1 = "Foo";
    std::string s2 = "Gaz";
    s2 = s1;
    s2[0] = 'B';
    std::cout << s1 << std::endl;
}
\end{lstlisting}

what might happen if you try to compile, link and run this program?

\begin{note}
\st

\begin{tiny}
I get:
\begin{verbatim}
g++ -Wall scratch.cpp && ./a.out
Foo
\end{verbatim}

Just for curiosity, what if you try?
\begin{verbatim}
#include <iostream>
#include <string>

int main() {
    char s1[] = "Foo";
    char s2[] = "Gaz";
    s2 = s1;
    s2[0] = 'B';
    std::cout << s1 << std::endl;
}
\end{verbatim}

Then I get:
\begin{verbatim}
g++ -Wall scratch.cpp && ./a.out
scratch.cpp: In function 'int main()':
scratch.cpp:7: error: ISO C++ forbids assignment of arrays
scratch.cpp:7: confused by earlier errors, bailing out
\end{verbatim}
(same is true for C99, C89)

And of course, if you attempt 'char *' then you get a runtime error.
\end{tiny}

\end{note}

\end{slide}

%%%%%

\renewcommand\st{�x.x, title}
\begin{slide}

\begin{lstlisting}
#include <iostream>
#include <string>

int main() {
    std::string s1 = "abbcccde";
    std::string::size_type p = s1.rfind("cc");
    s1.replace(p, 2, "XXX");
    std::string s2 = s1.substr(3, -2);
    std::cout << s2 << std::endl;
}
\end{lstlisting}

what might happen if you try to compile, link and run this program?

\begin{note}
\st

\begin{tiny}
I get:
\begin{verbatim}
g++ scratch.cpp && ./a.out
scratch.cpp: In function 'int main()':
scratch.cpp:8: warning: passing negative value '-0x00000000000000002' 
  for argument 2 to 'std::basic_string<_CharT, _Traits, _Alloc> 
  std::basic_string<_CharT, _Traits, _Alloc>::substr(typename _Alloc::size_type, 
  typename _Alloc::size_type) const [with _CharT = char, _Traits 
  = std::char_traits<char>, _Alloc = std::allocator<char>]'
cXXXde
\end{verbatim}
\end{tiny}

What would the value of 'p' be if the substring "cc" was not found? std::string::npos which is 4294967295 on my machine.
\end{note}

\end{slide}

%%%%%

\renewcommand\st{�x.x, title}
\begin{slide}

\begin{lstlisting}
#include <iostream>

int main() {
    int a = 4;
    int b = 2;
    std::clog << a << b;
}
\end{lstlisting}

what might happen if you try to compile, link and run this program?

\begin{note}
\st

\begin{tiny}

I get:
\begin{verbatim}
42
\end{verbatim}

What is the difference between cerr and clog?

Why was '<<' and '>>' chosen?

[p607] '<<' and '>>' was chosen also because they bind the right way, that is (cout << a) << b rather than cout << (a << b);


\end{tiny}

\end{note}

\end{slide}

%%%%%

\renewcommand\st{�x.x, title}
\begin{slide}

\begin{lstlisting}
#include <iostream>

struct A {
    virtual std::ostream & put(std::ostream &) const = 0;
};

struct B : A {
    std::ostream & put(std::ostream & s) const { return s << 'B'; }
};

std::ostream & operator<<(std::ostream & s, const A & a) {
    return a.put(s);
}
    
int main() {
    B b;
    std::cout << b << std::endl;
}
\end{lstlisting}

what might happen if you try to compile, link and run this program?

\begin{note}
\st

\begin{tiny}
I get:
\begin{verbatim}
g++ scratch.cpp && ./a.out
B
\end{verbatim}

[p612] This is a fine way to print out objects for which only a base class is known. 

\end{tiny}

\end{note}

\end{slide}

%%%%%

\renewcommand\st{�x.x, title}
\begin{slide}

\begin{lstlisting}
#include <iostream>

int main() {
    double pi = 3.14159265358979323846;
    std::cout << pi << std::endl;
    std::cout.precision(3);
    std::cout << pi << std::endl;
}
\end{lstlisting}

What does this print out? How could this code might look
like if we were using a stream manipulator instead?

\begin{note}
\st

\begin{tiny}
I get:
\begin{verbatim}
g++ scratch.cpp && ./a.out
3.14159
3.14
\end{verbatim}

You might consider using a manipulator instead. Eg,
\begin{verbatim}
#include <iostream>
#include <iomanip>

int main() {
    double pi = 3.14159265358979323846;
    std::cout << pi << std::endl;
    std::cout << std::setprecision(3) << pi << std::endl;
}
\end{verbatim}

[21.4.6.2 Standard I/O Manipulators, p633]

\end{tiny}

\end{note}

\end{slide}

%%%%%

\renewcommand\st{�x.x, title}
\begin{slide}

\begin{lstlisting}
#include <fstream>
#include <iostream>
#include <string>

int main() {

    std::string ostr = "This is a test of writing and reading from files";
    std::ofstream ofile("myfile.tmp");
    ofile << ostr;

    std::string istr;
    std::ifstream ifile("myfile.tmp");
    ifile >> istr;

    std::cout << istr;
}
\end{lstlisting}

What might happen if you try to compile, link and run this program? 

\begin{note}
\st

\begin{tiny}
I get nothing, beacuse ofile is not flushed before we start reading
from the file. Adding a ofile.flush() might help a bit. You can
also do an explisit ofile.close(), but this is also implicitly
done by the destructor [p639] so putting a scoping block around 
line 7-9 will work.

If you do that, then you will get 'This' printed out. Why? Because
the >> will read tokens delimited by whitespace. If you want to 
read a whole line you might consider:
\begin{verbatim}
#include <fstream>
#include <iostream>
#include <string>

int main() {

    std::string ostr = "This is a test of writing and reading from files";
    std::ofstream ofile("myfile.tmp");
    ofile << ostr;
    ofile.close();
    
    std::string istr;
    std::ifstream ifile("myfile.tmp");
    std::getline(ifile,istr);

    std::cout << istr;
}
\end{verbatim}
which might give:
\begin{verbatim}
g++ -Wall scratch.cpp && ./a.out
This is a test of writing and reading from files
\end{verbatim}

\end{tiny}

\end{note}

\end{slide}

%%%%%

\renewcommand\st{�x.x, title}
\begin{slide}

\begin{lstlisting}
#include <iostream>
#include <limits>

int main() {
    
    std::cout << std::numeric_limits<char>::digits << std::endl;
    
    std::cout << std::numeric_limits<int>::digits << std::endl;
    std::cout << std::numeric_limits<int>::max() << std::endl;
    std::cout << std::numeric_limits<int>::min() << std::endl;
}
\end{lstlisting}

what might happen if you try to compile, link and run this program?

\begin{note}
\st

\begin{tiny}
I get:
\begin{verbatim}
g++ -Wall scratch.cpp && ./a.out
7
31
2147483647
-2147483648
\end{verbatim}
or:
\begin{verbatim}
g++ -funsigned-char -Wall scratch.cpp && ./a.out
8
31
2147483647
-2147483648
\end{verbatim}
\end{tiny}

\end{note}

\end{slide}

%%%%%

\renewcommand\st{�x.x, title}
\begin{slide}

\begin{lstlisting}
#include <iostream>
#include <valarray>

double f(double d) {
    return d + 1;
}

int main() {
    const double a[] = {1.23, -4.54, 0.48, -1};
    const double b[] = {1, 0, 0, 1};

    std::valarray<double> va(a,4);
    std::valarray<double> vb(b,4);
    std::valarray<double> vc = va * vb;
    vc *= 2;
    std::valarray<double> vd = vc.apply(f);

    for (size_t i=-0; i<vd.size(); i++) 
        std::cout << vd[i] << " ";
}
\end{lstlisting}

what might happen if you try to compile, link and run this program?

\begin{note}
\st

\begin{tiny}

I get:
\begin{verbatim}
g++ -Wall scratch.cpp && ./a.out
3.46 1 1 -1 
\end{verbatim}
\end{tiny}

[p662] valarray - designed specifically for speed of the usual numeric vector operations.

\end{note}

\end{slide}

%%%%%
%% 
%% \renewcommand\st{�x.x, title}
%% \begin{slide}
%% 
%% \begin{lstlisting}
%% 
%% \end{lstlisting}
%% 
%% what might happen if you try to compile, link and run this program?
%% 
%% \begin{note}
%% \st
%% 
%% \begin{tiny}
%% \begin{verbatim}
%% 
%% \end{verbatim}
%% \end{tiny}
%% 
%% \end{note}
%% 
%% \end{slide}

\end{document}
